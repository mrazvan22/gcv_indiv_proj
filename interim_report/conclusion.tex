\chapter{Conclusion}

% Summary of achievements and Future work

This interim report offers a clear perspective on the work done on this project
so far. I have successfully developed the mathematical model for the GCV
signature and I have implemented in an efficient manner the core algorithms
necessary for the project. I have also performed some early experiments that
showed some interesting properties of the metric. 

Nonetheless, I also encountered several problems that I could not really
foresee, which delayed the usual progress of the project. For instance, right
at the very beginning I when I was working on the mathematical model, I could
not mathematically find what the maximum possible number of graphlets of a
particular type in a graph of N nodes. After I finished the mathematical model,
I also had to spend some time reformatting the graphlet counting function that I
was given. Although it took me a while to solve these tasks, I believe it was
good that I tackled them in this manner, as it helped me better understand the
nature of the project I was working on, both from a mathematical and a
computational perspective.
 
\section{Future work}

As I have previously mentioned, our future work mainly consists in performing
more experiments with the GCV signature in order to find out its properties.
Our group is fairly confident that we will get a lot of interesting results. We
are therefore planning to publish one paper with these results in a
Bioinformatics journal.

One other idea we could explore is to try to derive several related metrics
using different normalisation procedures or even combine the GCV with the older
GDV metric. This could allow us to effectively use the power of both metrics at
the same time. 

Another idea that I had while discussing with Zoran, one of Dr. Przulj's
collaborators, was to find out how important each of the elements from the
\emph{Graphlet Cluster Vector} was by assigning a weight to each of them. Using
some machine learning techniques or linear regression, some optimal weights
could be derived which would make the signatures more efficient when comparing
them with each other or when calculating the \emph{Relative Cluster Frequency
Distance}. Unfortunately, these tasks are outside the scope of our project, as
they can take a few months to complete and I wouldn't have enough time to finish
them by July. 

Finally, we hope that our work will help us better understand local properties
of biological or economic networks and that it will make an impact in the
Bioinformatics community. Ultimately, network analysis is a never-ending task:
one can always find better ways to explain phenomena or behaviour, and even
more, as this phenomena or behaviour changes over time, new models need to be
developed that model then as closely as possible.
