\chapter{Introduction}
%(1-3 pages). For the interim report this section should be a short, succinct,
%summary of
%the project’s main objectives. Some of this material may be re-usable in your
%final report, but the chances are that your final introduction will be quite
%different.  You are therefore advised to keep this part of the interim report
%short, focusing on the following questions: What is the problem, why is it
%interesting and what’s your main idea for solving it?  (DON'T use those three
%questions as subheadings however!  The answers should emerge from what you
%write.)

\section{Abstract}

Large biological data that has been made available in the past 20 years is
infeasible for the humans to manually analyse. Several algorithms and analysis
techniques have been introduced in order to process such data and extract
critical information from biological networks. We have developed a novel metric
that compares the local topological structure around a node belonging to such a
network. When applied to biological networks such as protein-protein interaction
networks or metabolic networks, this could tell us whether two proteins have a
similar function or even identify key proteins in the metabolic chain. Moreover,
it can also help us compare whole networks together and find out how similar or
different they are with respect to certain properties. We hope that our
technique would help scientists better understand biological and
economic networks.

\section{Motivation}

Over the past few decades, major advancements in Genomics and Molecular
Research technologies have made available large amounts of biological data that
can help us better understand molecular processes. A large set of the data is
in the form of biological networks, such as protein-protein interaction
networks or metabolic networks. Analysing these networks can help scientists
better understand the molecular interactions present in our body and can help
aid drug discovery or treatment of various disorders. However, as these
networks are too large to be analysed, computers are being extensively used to
find patterns in the data. Bioinformatics and Computational Biology are
therefore the two emerging fields that use computers to analyse biological
information.

The past two or three decades have been an exciting time for Bioinformatics.
From the computational side, several algorithms that align DNA sequences have
been developed, such as BLAST\cite{altschul1990basic} or
Smith-Waterman\cite{smith1981identification}. These algorithms can not
only quickly analyse large amounts of genetic data, but they can also be
parallelised on a cluster of computers. 

From a biological point of view, sequencing the human genome became reality for
the
first time through the Human Genome Project\cite{watson1990human}, a large
collaboration of research
institutes around the world. A project such as the Human Genome Project, as
well as others have produced large amounts of network data that is waiting to
be analised. Nowadays, there is a clear need of computer scientists and
bioinformaticians to analyse these data and collaborate with biologists and
chemists. 

Our project presents a novel method of comparing biological networks and
individual nodes, which can also be used to align networks or asses which
random model fits the real data best. We believe that exploring new ways to
analyse biological networks will help us shed a new light on complex biological
processes. This could further lead to an increased understanding of diseases
such as cancer or cardiovascular disorders that are leading causes of death
worldwide\cite{jemal2008cancer}\cite{world2004annex}. 

It is the case that when analysing networks of any kind, scientists generally
look at global properties of networks, such as the degree distribution,
average diameter or node centralities. These have so far been
successfully used to identify enzymes or reactions that are crucial for
the survival of organisms\cite{rahman2006observing}, model drug design
trends\cite{yildirim2007drug} or modeling the world-wide airport
network\cite{guimera2004modeling} However, not that much research has been
done in exploring local properties of the network, that describe how a node
interacts with its neighbourhood. Our technique of comparing nodes is based on
local information, and it can thus be used to complement other established
techniques.

Our technique builds upon work done by the Przulj group, which has developed a
metric that is based on graphlets (i.e. small
graphs) \cite{prvzulj2007biological}. Their research has shown that the
this metric has been successfully used to fit random network models to real
world networks\cite{prvzulj2004modeling}, uncover biological network
function\cite{milenkoviae2008uncovering} and topologically align
networks\cite{kuchaiev2010topological}. Our aim is to generalise this metric
and use it help discover new biological functions. Given these previous results,
we believe our technique will also be successfully used to fit
random network models to real networks and topologically align networks.

\section{Objectives}

We would like to explore a new way of analysing biological networks by
looking at local topology and structures around individual nodes. This
would be performed by the means of a new metric called \textbf{Graphlet Cluster
Vector} (GCV) that we will develop, which builds on previous work done by Natasa
Przulj\cite{prvzulj2007biological}. This metric measures the number of
graphlets (i.e. small graphs with up to 5 nodes) that are found in the
neighbouring subgraph of a node. For such a node, this metric quantifies the
type of interaction its neighbours have with each other. As it will be 
explained later on, one can see it as a generalisation of the clustering coefficient.

Our first objective is to implement a fast algorithm that calculates the new
metric for every single node in a given network. As some of the networks are
large and the computation is quite intensive, we are also planning to paralellise
the algorithm in order to enable it to run on multiple cores or even on a cluster of
servers.

Our main objectives are to explore the properties of this novel metric and
used it to compare not only individual nodes, but also whole networks together.
First of all, we will be trying to find out what are the differences between the
GCV and an older GDV metric that was developed by Tijana and
Przulj\cite{milenkoviae2008uncovering}. This could help us find certain areas
where it could be more efficient to use the new metric for biological analysis. 

Afterwards, the next objective would be to compute an average GCV signature for
the entire network, by averaging the individual GCV of every single node in it.
This would enable us to perform early comparisons of the GCV between different
network types: 
\begin{itemize}
 \item Real networks
 \begin{enumerate}
    \item Protein-protein interaction networks
    \item Metabolic networks
    \item Trade networks
  \end{enumerate}
 \item Random networks
 \begin{enumerate}
    \item Erdos-Renyi\cite{erdHos1959random}
    \item Erdos-Renyi (arbitrary degree distribution)
    \item Geometric networks\cite{penrose2003random}
    \item Barabasi-Albert (preferential attachment)\cite{barabasi1999emergence}
    \item Stickiness index-based\cite{prvzulj2006modelling}
  \end{enumerate}
\end{itemize}

Moreover, previous research has already shown that the old GDV metric has been
successfully
used to fit random network models to real world
networks\cite{prvzulj2004modeling},  uncover biological network
function\cite{milenkoviae2008uncovering} and topologically align
networks\cite{kuchaiev2010topological}. Therefore, our objective is to asses
the performance of the new GCV metric on such problems and compare them with the
results of the GDV metric. We are thus mainly interested to find out how to
gather biological insights from the data we have. Apart from biological data,
we would like to see how the metric performs on economic networks, which have
a different structure. We will also be interested to evaluate a few other
properties, such as robustness to noise.

Other extensions that could possibly be done are as follows:
\begin{itemize}
\item use the new GCV metric to get new insights from biological data
\item explore the injectivity of the metric, by answering the following
questions: Are there 2 nodes belonging to the same biological network that can
have the same GCV result, even if the local structure around them is different?
What is the probability of such a clash occurring?
\end{itemize}
