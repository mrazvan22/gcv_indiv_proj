\chapter{Project Plan}
%(1-2 pages). You should explain what needs to be done in order to complete the
%project and
%roughly what you expect the timetable to be. Don’t forget to include the
%project write-up, as this is a major part of the exercise. It’s important to
%identify key milestones and also fall-back positions, in case you run out of
%time.  You should also identify what extensions could be added if time
%permits. The plan should be complete and should include those parts that you
%have already addressed (make it clear how far you have progressed at the time
%of writing). This material will not appear in the final report.

\section{Main Objectives}

The aim of the project is to implement this \emph{Graphlet Cluster Vector} (GCV)
signature and explore its properties. We are interested in finding out to what
degree it is different from the older \emph{Graphlet Degree Vector} (GDV)
developed by Przulj et al\cite{prvzulj2007biological}. The main objectives that
we have set are the following:
\begin{enumerate}
 \item Implement the algorithm for calculating the GCV signature
 \item Parallelise it in order to make it speed up analysis. This is
particularly important when analysing large networks, such as the
protein-protein interaction networks
 \item Implement an algorithm to compute the average network GCV
 \item Compute the average GCV of different networks and compare them with each
other. Find out whether this metric can be used to distinguish between them.
 \item Generate random networks from different models (Erdos-Renyi, Geometric,
Barabasi-Albert, Stickiness based) using the real networks as models.
 \item Calculate the GCV of the generated random networks and compare the
results against the real networks. Compute the Graphlet frequency agreement
and find out which random network fits the real data best according to the GCV
metric.
 \item Generate more instances of the random networks (around 10 for each
model), compute their GCV and then plot the standard deviation of the GCV
\end{enumerate}

\subsection{Extensions}
Accomplishing all the these objectives would be enough to get a clear
understanding of the properties of the GCV signature. If time allows, a more
comprehensive analysis can be peformed by doing several extras:
\begin{enumerate}
 \item Use this metric to align two networks with each other using the GRAAL
algorithm\cite{memivsevic2012c}
 \item Explore the injectivity of the metric, by answering the following
questions: Are there 2 nodes belonging to the same biological network that can
have the same GCV result, even if the local structure around them is different?
What is the probability of such a clash occuring?
 \item Find out the evolution of world trade over time by analysing different
trade networks from successive years. We can compute the GCV of the trade
networks from successive years and find out what kind of change in patterns we
get when:
 \begin{itemize}
  \item An economic crisis occurs: 2007 submortgage economic crisis, Eurozone
crisis
  \item A major global political and economic shift occurs: Velvet Revolutions
of Eastern Europe, Arab Spring
  \item Free trade areas are created
 \end{itemize}

\end{enumerate}

\section{Work done so far}

I have started working on this project in April last year, immediately after we
started our industrial placements. However, I could not dedicate too much time
on it, for I was working full time for my industrial placement. Nonetheless, I
managed to do some background reading and to explore the mathematical model
behind the graphlet counting procedure. More precisely, I spent a considerable
ammount of time trying to devise a more precise normalisation procedure for the
Graphlet Cluster Vector. More precisely, I tried to find out what is the
maximum number of graphlets of each type one could have in a graph of n
nodes, in order to use these numbers for normalisation. However, this turned out
to be a very complex mathematical problem that I could not practically solve.
The reason for this was because it is really hard to:
\begin{itemize}
 \item find out what the ideal network architecture is in which one can get
the maximum number of graphlets of a particular type
 \item mathematically find out the maximum number of graphlets in such an ideal
architecture as a formula of N, the number of nodes in the network.
\end{itemize}

Therefore, we have decided to only perform a simple normalisation, by dividing
the frequencies of each graphlet types against the sum of all frequencies.

So far, I have managed to do the following:
\begin{itemize}
 \item Implement the algorithm which calculates the GCV for a particular node in
the network
 \item Parallelise the computation over several cores
 \item Implement an algorithm to compute the average network GCV
 \item Compute the average GCV of different networks and compare them with each
other. 
 \item Generate random networks from different models (Erdos-Renyi, Geometric,
Barabasi-Albert, Stickiness based) using the real networks as models.
 \item Calculate the GCV of the generated random networks.

\end{itemize}

\section{Future milestones}

The milestones for the next key tasks are described below:
\begin{itemize}
 \item \textbf{30 February}: Calculate the Relative Cluster Frequency Distance
(RCFD) between random networks and real networks
 \item \textbf{30 March}: Generate 10 network instances for each random model.
Calculate the average GCV and plot it along with the standard deviation.
 \item \textbf{14 April}: Calculate the Pearson's correlation coefficient
between pairs of column vectors of the GCV matrix of a network. Derive a 30x30
matrix of correlation pairs that is used as a global signature of the network.
Given 2 such matrices, we can compare then and find out how similar two
networks are.
 \item \textbf{15 May}: Use the GCV and the correlation matrix to derive
biological or economic insights from the network data. For example, we can
analyse gene networks and transfer function from one gene to another by
performing alignment.
 \item \textbf{30 May}: Use the GCV to align two networks with each other
using
the GRAAL algorithm\cite{memivsevic2012c}. Compare the performance with that of
the old GDV signature.
 \item \textbf{15 June}: Perform a time-dependent analysis of trade networks.
Find out the RCFD between trade networks at different points in time.
 \item \textbf{27 June}: Write the final report


\end{itemize}


